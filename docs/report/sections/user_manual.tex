\documentclass[../report.tex]{subfiles}
\begin{document}   

% User manual
\subsection{Software Setup Instructions}

For the simple case where no API keys need to be changed, setting up and running UltraCast is as simple as running:

\begin{verbatim}
    # cd to the root directory of the git repo
    ./start.sh
\end{verbatim}
%
This script will: 
\begin{itemize}
    \item Create a python venv for the backend
    \item Install all required python packages in the venv
    \item Install all npm packages that are required for the frontend
    \item Launch the backend webserver (if the --local flag is used)
    \item Launch the frontend
    \item Open ultracast in your browser (this may not work on Vlab)
\end{itemize}

UltraCast can be run using either a local or remote GraphQL endpoint.
The remote GraphQL endpoint is preffered due to lower latency (it must make successive MongoDB database operations which can be slow if network latency is high).

To run UltraCast using a local backend webserver:

\begin{verbatim}
    # cd to the root directory of the git repo
    ./start.sh --local
\end{verbatim}

Once the webserver is installed and running you will see a message like:

\begin{verbatim}
    Serving on port 4000
\end{verbatim}

You can then navigate to UltraCast in a webbrowser at \verb|localhost:4000|

\subsection{Configuration}

The following external services are used and their IP addresses and/or API keys will need to be set in configuration files:

\begin{itemize}
    \item Algolia
    \item MongoDB Instance (hosted in a cloud container e.g. Amazon EC2)
    \item S3 Bucket
    \item Backend GraphQL endpoint (if not hosted on local machine)
\end{itemize}

Some of these variables need to be set for the frontend and some for the backend.

\subsubsection{Backend Configuration}

The backend is configured by using python files which set various configuration variables. These include options including:
\begin{itemize}
    \item The IP address of the MonogDB instance
    \item The MonogDB database
    \item Flask secret keys (for encryption)
    \item Algolia API key and user
\end{itemize}
A full list of the variables that can be set is in \verb|backend/config/default_settings.py|.
%
Any variables that are not set are defaulted to the value in \verb|backend/config/default_settings.py|.
You can override these settings by writing a new python file and setting the environment variable \verb|ULTRACAST_BACKEND_SETTINGS| to be the real path of this file. 
For example if the settings file is at \verb|~/ultracast_settings.py|, you could do:

\begin{verbatim}
    export ULTRACAST_BACKEND_SETTINGS=$(realpath ~/ultracast_settings.py)
    bash backend/start.sh
\end{verbatim}

\subsubsection{Frontend Configuration}

The frontend can be configured by editing the file \verb|frontend/src/api/config.js| Here you can set options including:
\begin{itemize}
    \item The backend GraphQL endpoint to use
    \item Algolia API key and user
\end{itemize}

\subsection{Logging}

When \verb|./start.sh| or \verb|backend/start_production.sh| are used to launch the backend, by default backend logs will be recorded to \verb|backend/logs.txt|.
This behaviour can be modified by changing the err-log option in \verb|backend/backend_app.py|.
When the backend is run in development mode, logs are printed to the terminal.

Frontend logs can be accessed via the webbrowser console.

\subsection{Site Usage and Functionality Guide}

% Keep in mind this section is worth 50%

% Extra subsection per story / functionality? Might group some together

% Each person has to demo a set of features for the final demo, it might make sense for each
% person to add to the manual for the features they are demoing.
%      - NOTE: We should agree of how these instrustions are laid out (+ how images are displayed) tho so things are uniform
%      - Might have one person do it earlier / set the standard for everyone else to follow
\end{document}

